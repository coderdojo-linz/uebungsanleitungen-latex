\documentclass{article}

\usepackage[pdfauthor={CoderDojo Linz},
            pdftitle={Bubble Sorter in TypeScript}]
            {hyperref}



\newcommand{\footertitle}{Bubble sorter in TypeScript}
% settings.tex
\usepackage[
    a4paper, 
    top=3cm,
    left=1cm,
    right=1cm,
    bottom=2cm
]{geometry}

\usepackage{fontspec}
\usepackage{graphicx} % package to insert images
\usepackage{hyperref} % for hyperlinks
\usepackage{fancyhdr}
\usepackage[ngerman]{babel}
\usepackage{wrapfig}
\usepackage{enumitem}
\usepackage{titlesec} % For section customization
\usepackage{xcolor} % For color
\usepackage{sectsty} % For sections
\usepackage{ragged2e}

\setmainfont{Carlito}

% Fancyhdr setup
\fancypagestyle{defaultpagestyle}{
    \fancyhf{} % Clear all headers and footers
    \fancyhead[C]{\includegraphics[width=5cm]{images/CoderDojo_Logo.png}} 
    \renewcommand{\headrulewidth}{0pt} % Remove header line
    \renewcommand{\footrulewidth}{0pt} % Remove footer line
    \fancyfoot[L]{\footertitle} % Left footer
    \fancyfoot[R]{Seite \thepage} % Right footer with page number
}

\newcommand{\SectionDesign}[4]{
    \noindent
    \csname #1*\endcsname{\textcolor[HTML]{1E90FF}{\fontsize{#2pt}{#3pt}\selectfont #4}}
}

\newcommand{\TextAndImage}[5][{}]{
    \fontsize{11pt}{16pt}\selectfont
    \noindent
    \begin{minipage}[c]{#4\textwidth}
    \RaggedRight
    #2 % First parameter: text
    \end{minipage}
    \hfill
    \begin{minipage}[c]{#5\textwidth}
    \includegraphics[width=\textwidth, #1]{#3} % Second parameter: image file name
    \end{minipage}
}

\newcommand{\ImageAndText}[2]{
    \fontsize{16pt}{24pt}\selectfont
    \noindent
    \begin{minipage}[c]{0.45\textwidth}
        \includegraphics[width=\textwidth]{#1} % Second parameter: image file name
    \end{minipage}
    \hfill % Fills the space between the minipages
    \begin{minipage}[c]{0.45\textwidth}
        \centering
        #2 % First parameter: text
    \end{minipage}
}

\graphicspath{{images/}}

\begin{document}
    \pagestyle{defaultpagestyle}

    \SectionDesign{section}{24}{24}{\textbf{Bubble Sorter mit TypeScript}}
    \vspace{1cm}

       \ImageAndText{MainPic.png}{
    \centering
    Heute programmieren wir unseren Bubble-Sorter in Typescript! Hiermit kannst du gut in Textuelle Programmierung einsteigen.
    }{0.6}{0.3}{16}{24}

    \vspace{1cm}
    \TextDesign{
    Heute programmieren wir eine TypeScript Übung namens Bubble Sorter. Hierbei kannst du deine TypeScript Kenntnisse verbessern.}{0.6}{0.3}{16}{24}
    
    \vspace{2cm}
    \SectionDesign{subsection}{18}{24}{\textbf{Ziel der Übung}}
    \vspace{0.5cm}

    \TextDesign{
    Heute wollen wir gemeinsam einen Bubble-Sorter in TypeScript programmieren. Hierbei geht es darum langsam in Textuelle Programmierung einzusteigen. Du solltest die Übung davor in Scratch gemacht haben! Gleich darunter findest du deinen Starter-Code}
\begin{itemize}
        \item \url{https://stackblitz.com/edit/coderdojo-bubbles-cvzz1t?file=index.ts,helpers.ts}

 
    \end{itemize}
\newpage


          \vspace{0.5cm}

  
\end{document}