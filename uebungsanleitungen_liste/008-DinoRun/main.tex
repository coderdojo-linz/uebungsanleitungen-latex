\documentclass{article}

\usepackage[pdfauthor={CoderDojo Linz},
            pdftitle={dino-run}]
            {hyperref}

\newcommand{\footertitle}{Dino Run}
% settings.tex
\usepackage[
    a4paper, 
    top=3cm,
    left=1cm,
    right=1cm,
    bottom=2cm
]{geometry}

\usepackage{fontspec}
\usepackage{graphicx} % package to insert images
\usepackage{hyperref} % for hyperlinks
\usepackage{fancyhdr}
\usepackage[ngerman]{babel}
\usepackage{wrapfig}
\usepackage{enumitem}
\usepackage{titlesec} % For section customization
\usepackage{xcolor} % For color
\usepackage{sectsty} % For sections
\usepackage{ragged2e}

\setmainfont{Carlito}

% Fancyhdr setup
\fancypagestyle{defaultpagestyle}{
    \fancyhf{} % Clear all headers and footers
    \fancyhead[C]{\includegraphics[width=5cm]{images/CoderDojo_Logo.png}} 
    \renewcommand{\headrulewidth}{0pt} % Remove header line
    \renewcommand{\footrulewidth}{0pt} % Remove footer line
    \fancyfoot[L]{\footertitle} % Left footer
    \fancyfoot[R]{Seite \thepage} % Right footer with page number
}

\newcommand{\SectionDesign}[4]{
    \noindent
    \csname #1*\endcsname{\textcolor[HTML]{1E90FF}{\fontsize{#2pt}{#3pt}\selectfont #4}}
}

\newcommand{\TextAndImage}[5][{}]{
    \fontsize{11pt}{16pt}\selectfont
    \noindent
    \begin{minipage}[c]{#4\textwidth}
    \RaggedRight
    #2 % First parameter: text
    \end{minipage}
    \hfill
    \begin{minipage}[c]{#5\textwidth}
    \includegraphics[width=\textwidth, #1]{#3} % Second parameter: image file name
    \end{minipage}
}

\newcommand{\ImageAndText}[2]{
    \fontsize{16pt}{24pt}\selectfont
    \noindent
    \begin{minipage}[c]{0.45\textwidth}
        \includegraphics[width=\textwidth]{#1} % Second parameter: image file name
    \end{minipage}
    \hfill % Fills the space between the minipages
    \begin{minipage}[c]{0.45\textwidth}
        \centering
        #2 % First parameter: text
    \end{minipage}
}

\graphicspath{{images/}}

\begin{document}
    \pagestyle{defaultpagestyle}

    \SectionDesign{section}{24}{24}{\textbf{Dino Run}}

    \SectionDesign{subsection}{18}{24}{\textbf{Einleitung}}
    
    \TextDesign{
    Heute werden wir das berüchtigte \textit{T-Rex Jump \& Run} von der Offline-Seite von Google Chrome nachbauen. Wir werden uns bei der Umsetzung auf das tolle Youtube-Tutorial von Ania Kubów stützen. Falls du es noch nicht getan hast, schau dir ihren Youtube-Kanal an. Sie hat noch mehr lustige Tutorials für die Programmierung von lustigen Spielen mit JavaScript.}

    \SectionDesign{subsection}{18}{24}{\textbf{Entwicklungsumgebung laden}}
    
    \TextAndImage{
     Als erstes müssen wir die \textit{Entwicklungsumgebung} laden. Du kannst direkt im Browser auf Stackblitz arbeiten oder eine lokale Umgebung wie Visual Studio Code verwenden. Verwendest du Stackblitz, dann müssen wir ein neues Projekt starten, wir verwenden dazu das Template \textit{Static (HTML/CSS/JS)}. Wir brauchen drei Dateien, die nebeneinander liegen: index.html, dino.css und dino.js.
    }{new-static.png}{0.55}{0.40}{11}{16}

    \vspace{0.5cm}

     \TextDesign{
     Der Einstiegspunkt der Website ist die \textit{index.html}, in dieser Datei müssen wir Verweise auf unser Stylesheet \textit{dino.css} und unsere JavaScript-Datei \textit{dino.js} hinzufügen. Wir fügen sie auf der Grundlage ihres relativen Pfads zum Speicherort der Datei \textit{index.html} ein. Falls sich diese Dateien im selben Ordner befinden, benötigen wir nur den Dateinamen.
     }

    \begin{lstlisting}[language=HTML]
<head>
  <link rel="stylesheet" href="dino.css"></link>
  <script src="dino.js" charset="utf-8"></script>
</head>
\end{lstlisting}

    \TextDesign{
    Um zu testen, ob unsere Dateien korrekt verlinkt sind, fügen wir unserer Javascript- und CSS-Datei etwas willkürlichen Code hinzu. Wir fügen einen alert(„Hallo T-Rex“) in die Javascript-Datei und einen body { background-color: hotpink; } in unsere css-Datei ein. Wenn wir unsere index.html-Datei im Browser öffnen, sollten wir einen rosafarbenen Hintergrund und eine Meldung sehen, die erscheint.
    }

    \vspace{0.5cm}

    \newpage

    \SectionDesign{subsection}{18}{24}{\textbf{T-Rex}}

    \SectionDesign{subsection}{12}{18}{Hinzufügen unseres Dinos}

    \TextDesign{
Zunächst fügen wir in der Datei index.html einige HTML-Tags zu unserer Website hinzu (⚠️ innerhalb des Body-Tags!). Wir brauchen einen Container-Tag, der alle unsere Spielelemente enthält - wie unseren Dino und alle Hindernisse. Und wir werden einen weiteren Container für unseren Dino hinzufügen. Wir werden div-Tags für unsere Container verwenden.
    }

\begin{lstlisting}[language=HTML]
    <div class="grid">
      <div class="dino"></div>
    </div>
\end{lstlisting} 


\TextDesign{Wenn wir die Seite jetzt speichern und neu laden, sind keine Änderungen sichtbar, aber wir sollten diese Elemente sehen, wenn wir das HTML-Markup untersuchen. Um unseren Dino sichtbar zu machen, fügen wir in unserer dino.css einige CSS-Befehle ein. Beginnen wir damit, seine Größe zu definieren und ihm eine Hintergrundfarbe zu geben, damit er sichtbar wird.}

\begin{lstlisting}
.dino {
  width: 60px;
  height: 60px;
  background-color: red;
}
\end{lstlisting}

\vspace{0.5cm}

\TextDesign{
⚠️ .dino ist ein CSS-Klassenselektor. Er wendet die Styles auf alle Elemente an, die die CSS-Klasse „dino“ haben. Du kannst mehr über CSS-Selektoren auf W3Schools lesen.
}

\TextDesign{
Jetzt werden wir ein rotes Quadrat in der oberen linken Ecke unserer Webseite sehen. Eigentlich möchten wir es in der unteren linken Ecke sehen, also ändern wir die Positionierung auf absolut, indem wir die folgenden zwei Zeilen hinzufügen.
}

\begin{lstlisting}
.dino {
  position: absolute;
  bottom: 0px;
}
\end{lstlisting}
  
\end{document}